\documentclass{bmstu}

\begin{document}

\makereporttitle
{Информатика и системы управления (ИУ)}
{Программное обеспечение ЭВМ и информационные технологии (ИУ7)}
{\textbf{2}}
{Конструирование компиляторов}
{Преобразования грамматик}
{6}
{ИУ7-22М}
{К.Э. Ковалец}
{А.А. Ступников}


\setcounter{page}{2}
% \renewcommand{\contentsname}{Содержание} 
% \tableofcontents


\chapter{Выполнение лабораторной работы}

\section{Общий вариант для всех: Устранение левой рекурсии.}

\textbf{Определение.} Нетерминал $A$ КС-грамматики $G = (N, \Sigma, P, S)$ называется рекурсивыным, если $A =>+ \alpha A \beta$ для некоторых $\alpha$ и $\beta$. Если $\alpha = \epsilon$, то $A$ называется леворекурсивным. Аналогично, если $\beta = \epsilon$, то $А$ называется праворекурсивным. Грамматика, имеющая хотя бы один леворекурсивный нетерминал, называется леворекурсивной. Аналогично определяется праворекурсивная грамматика. Грамматика, в которой все нетерминалы, кроме, быть может, начального символа, рекурсивные, называется рекурсивной.

Некоторые из алгоритмов разбора не могут работать с леворекурсивными грамматиками. Можно показать, что каждый КС-язык определяется хотя бы одной не леворекурсивной грамматикой.

Постройте программу, которая в качестве входа принимает приведенную КС-грамматику $G = (N, \Sigma, P, S)$ и преобразует ее в эквивалентную КС-грамматику $G'$ без левой рекурсии.

\section{Преобразование к нормальной форме Хомского.}

\textbf{Определение.} КС-грамматика $G = (N, \Sigma, P, S)$ называется грамматикой в \texttt{нормальной форме Хомского} (или в \texttt{бинарной нормальной форме}), если каждое правило из $Р$ имеет один из следующих видов:
\begin{enumerate}
    \item $A \rightarrow BC$, где $A$, $B$ и $C$ принадлежат $N$,
    \item $A \rightarrow a$, где $a \in \Sigma$,
    \item $S \rightarrow \epsilon$, если $\epsilon \in L(G)$, причем $S$ не встречается в правых частях правил.
\end{enumerate}

Можно показать, что каждый КС-язык порождается грамматикой в нормальной форме Хомского. Этот результат полезен в случаях, когда требуется простая форма представления КС-языка.

Постройте программу, которая в качестве входа принимает приведенную КС-грамматику $G = (N, \Sigma, P, S)$ и преобразует ее в эквивалентную КС-грамматику $G'$ в нормальной форме Хомского.





\clearpage
\section{Результаты работы программы}

Результаты работы программы по преобразованию грамматик приведены на рисунках \ref{img:left_recursion_1.png}--\ref{img:chomsky_form_3_res.png}.

\subsection{Устранение левой рекурсии}

\imgs{left_recursion_1.png}{h!}{0.52}{Исходная грамматика для удаления левой рекурсии (пример 1)}

\imgs{left_recursion_1_res.png}{h!}{0.52}{Грамматика после удаления левой рекурсии (пример 1)}

\imgs{left_recursion_2.png}{h!}{0.52}{Исходная грамматика для удаления левой рекурсии (пример 2)}

\imgs{left_recursion_2_res.png}{h!}{0.52}{Грамматика после удаления левой рекурсии (пример 2)}

\clearpage

\imgs{left_recursion_3.png}{h!}{0.52}{Исходная грамматика для удаления левой рекурсии (пример 3)}

\imgs{left_recursion_3_res.png}{h!}{0.52}{Грамматика после удаления левой рекурсии (пример 3)}

\subsection{Устранение левой факторизации}

\imgs{left_factorization.png}{h!}{0.52}{Исходная грамматика для удаления левой факторизации}

\imgs{left_factorization_res.png}{h!}{0.52}{Грамматика после удаления левой факторизации}

\clearpage

\subsection{Преобразование КС-грамматики к нормальной форме Хомского}

\imgs{chomsky_form_1.png}{h!}{0.52}{Исходная грамматика перед преобразованием к нормальной форме Хомского (пример 1)}

\imgs{chomsky_form_1_res.png}{h!}{0.52}{Грамматика после преобразования к нормальной форме Хомского (пример 1)}

\imgs{chomsky_form_2.png}{h!}{0.52}{Исходная грамматика перед преобразованием к нормальной форме Хомского (пример 2)}

\imgs{chomsky_form_2_res.png}{h!}{0.52}{Грамматика после преобразования к нормальной форме Хомского (пример 2)}

\clearpage

\imgs{chomsky_form_3.png}{h!}{0.52}{Исходная грамматика перед преобразованием к нормальной форме Хомского (пример 3)}

\imgs{chomsky_form_3_res.png}{h!}{0.52}{Грамматика после преобразования к нормальной форме Хомского (пример 3)}


\chapter{Контрольные вопросы}

\begin{enumerate}
    \item Как может быть определён формальный язык?
    \begin{enumerate}
        \item Простым перечислением слов, входящих в данный язык.
        \item Словами, порождёнными некоторой формальной грамматикой.
        \item Словами, порождёнными регулярным выражением.
        \item Словами, распознаваемыми некоторым конечным автоматом.
    \end{enumerate}
    \item Какими характеристиками определяется грамматика?
    \begin{enumerate}
        \item $\Sigma$ --- множество терминальных символов.
        \item $N$ --- множество нетерминальных символов.
        \item $P$ --- множество правил (слева --- непустая последовательность терминалов/нетерминалов, содержащая хотя бы один нетерминал, справа --- любая последовательность терминалов/нетерминалов).
        \item $S$ --- начальный символ из множества нетерминалов.
    \end{enumerate}
    \item Дайте описания грамматик по иерархии Хомского.
    \begin{enumerate}
        \item Регулярные.
        \item Контекстно-свободные.
        \item Контекстно-зависимые.
        \item Неограниченные.
    \end{enumerate}
    \item Какие абстрактные устройства используются для разбора грамматик?
    \begin{enumerate}
        \item Распознающие грамматики --- устройства (алгоритмы), которым на вход подается цепочка языка, а на выходе устройство печатает <<Да>>, если цепочка принадлежит языку, и <<Нет>> --- иначе.
    \end{enumerate}
    \item Оцените временную и емкостную сложность предложенного вам алгоритма.
    \begin{enumerate}
        \item Алгоритм удаления левой рекурсии
        \begin{itemize}
            \item $O(N^2)$ --- временная сложность;
            \item $O(N)$ --- ёмкостная сложность.
        \end{itemize}
    \end{enumerate}
\end{enumerate}


\chapter{Текст программы}

В листингах \ref{lst:main}--\ref{lst:color} представлен код программы.

\mylisting[python]{main.py}{firstline=1,lastline=100}{Основной модуль программы}{main}{}

\mylisting[python]{grammar.py}{firstline=1,lastline=322}{Модуль для преобразования грамматик}{grammar}{}

\mylisting[python]{color.py}{firstline=1,lastline=6}{Модуль с вариантами цветов при выводе сообщений в консоль}{color}{}

\end{document}
